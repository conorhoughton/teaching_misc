\documentclass[11pt,a4paper]{scrartcl}
\typearea{12}
\usepackage{graphicx}
\usepackage{pstricks}
\usepackage{amsmath}
\begin{document}


\section*{An introduction to programming}

To save time learning syntax we are going to look at a \textsl{block
  programming language} where the commands appear as block. Some of
you will have seen block programming before, in which case this is a
reminder, for some others this will be new. When you get used to a
programming language typing is, of course, quicker than moving blocks
around, but block programming languages are very convenient; this
programming environment is hosted online which is even more convenient
since it means we don't have to install anything. The site an be found
at \texttt{snap.berkeley.edu}. The notes guide you through some
example, the exercises are bulletpointed along the way.

\subsection*{Basic example}

Here is a simple programme for drawing a square; we will start with
this and try to make more complicated drawings.
\begin{center}
\includegraphics{basic_square.png}
\end{center}
Enter this and make sure it draws a square! One thing about this
program is that after it draws the square the arrow ends up pointing a
different direction to the direction it started in. 
\begin{itemize}
\item Can you fix this?
\end{itemize}

\subsection*{A loop}

Now, the annoying thing about this programme is having to move over
the same two commands again and again; this isn't just a problem
because it is boring, it also disguises the main point of the program,
doing the same thing four times. Programmes are best when they are
easy to interpret, so here we can make the programme simpler using a
repeat:
\begin{center}
\includegraphics{repeat.png}
\end{center}
\begin{itemize}
\item Can you use that to make the square drawing programme more succinct and readable? 
\item Can you make a programme to draw something that looks like a circle but going forward a tiny bit and turning again and again?
\end{itemize}

\subsection*{Variables}

Now, look at this programme
\begin{center}
\includegraphics{variable.png}
\end{center}
Instead of writing directly that it is to go 100 steps here we make a
\textsl{variable} called \texttt{x} and tell it to go \texttt{x}
steps. This is kind of pointless in this short program but we will see
soon how useful variables are. 
\begin{itemize}
\item Do the same to your square programme!
\end{itemize}
You will need to click the \lq{}Make a variable\rq{} button to add a
variable; add it for all sprites, we'll only be using one sprite at a
time in this class.

This programme does something slighly more useful with a variable, it
goes forward a certain number steps, turns, then goes forward half as
far as before. Now, rather than having to put in the number of steps,
then work out half and put that in, it uses a variable for the number
of steps and another one to calculate half the number. 
\begin{center}
\includegraphics{line_half_line.png}
\end{center}
\begin{itemize}
\item Try modifying your programme in a similar way so that it draws an
$n$-gon, notice that turning 90 degrees each time won't give an
$n$-gon, so you'll need to use a variable to instruct it to turn
$$
\theta=\frac{360}{n}
$$
degrees each time.
\end{itemize}

In this programme the variable is changed in the \textsl{loop} so the
line is shorter each time:
\begin{center}
\includegraphics{spiral.png}
\end{center}
giving a spiral
\begin{center}
\includegraphics{spiral_pic.png}
\end{center}
\begin{itemize}
\item Try modifying your programme in the same way so that you get smaller and smaller squares retreating into one corner, like this
\begin{center}
\includegraphics{repeat_squares_pic.png}
\end{center}
\item If you want to you can try playing with you programme a bit to give other patterns, like this
\begin{center}
\includegraphics{squares_spiral_pic.png}
\end{center}
\item Try modifying your circle programme to draw a round spiral.
\end{itemize}

\subsection*{If statement}

This next programme draws a star
\begin{center}
\includegraphics{star.png}
\end{center}
Input it and have a look and then try to understand it; it is intended
to illustrate the idea of conditional statements. There are two, first
the \textsl{repeat until}. This is similar to the repeat we used
before, but now, instead of repeating a set number of times, every
time it repeats it checks a condition, in this case \texttt{x=19}; if
the condition is true, it stops, otherwise it keeps repeating. In our
case one gets added to \texttt{x} each time, so eventually it will
stop. The other conditional statement is the \texttt{if . . . else}
statement. Because the star has two different angles we need two
different sorts of turns, in the if statement it does the first
possibility, turning 175 degrees if \texttt{x mod 2 = 0} and the other
otherwise. The meaning of \textsl{mod} is that it gies the remainder
after dividing, so \texttt{x mod 2} means the remainder after dividing
\texttt{x} by 2; this will be zero if \texttt{x} is even, one
otherwise. 

\begin{itemize}
\item You can mess with programme a bit, maybe changing the angles or
putting the whole thing in a loop to give something like this
\begin{center}
\includegraphics{rotating_star_pic.png}
\end{center}
\end{itemize}

Next have a look at this programme:
\begin{center}
\includegraphics{center_rays.png}
\end{center}
It also has a conditional statement, but the condition is something
outside the programme, in this case touching the edge.

\subsection*{Blocks}

Imagine you want to use the same commands a few times; in this
programme for example we draw a square, move over a bit and draw
another:
\begin{center}
\includegraphics{two_squares.png}
\end{center}
This is annoying, again, the extra typing, but also, as we discussed
before, it isn't as readable as it could be; each time you get to the
square drawing piece you need to work out what it is. Most programming
languages get around this with \textsl{functions} and the block
programming language we are using here has functions, it calls them
\textsl{blocks}. A block is a piece of code with a name. Here we will
make a block called square that draws a square: the block commands are
at the bottom of the variable menu:
\begin{center}
\includegraphics{two_squares_block_block.png}
\end{center}
This block draws a square, so our two square code becomes a bit
neater, quicker to input and easier to read:
\begin{center}
\includegraphics{two_squares_block.png}
\end{center}
You can make your blocks more flexible by adding \textsl{arguments};
these are variables that work inside the block that you can send from
the main programme, you make them by clicking the plus by the block
name.
\begin{center}
\includegraphics{two_squares_block_size_block.png}
\end{center}
and
\begin{center}
\includegraphics{two_squares_block_size_block.png}
\end{center}
draws the two squares different sizes. 

\begin{itemize}
\item Try writing a block to make a circle with the radius as the
  argument, remembering that:
$$
\mbox{circumference}=2\pi \mbox{radius}
$$
and $$2\pi / 360 \approx 0.017453$$.
\item Can you improve your circle block so that you also send the $x$ and $y$ location of the center of the circle?
\item Draw a snowman with a top hat and a carrot nose.
\end{itemize}


If anyone is finished with all of this maybe go on to this programme
\begin{center}
\includegraphics{koch.png}
\end{center}
with block
\begin{center}
\includegraphics{koch_block.png}
\end{center}
This programme shows \textsl{recursion}, the block called
\texttt{line} calls itself! Basically, if \texttt{iteration} is zero
then \texttt{line} draws an ordinary line, otherwise it draws a
special zigged line with each of the four segments of the zigged line
made up of calls to \texttt{line} but with \texttt{iteration} reduced
by one. This will be easier to see if you run the programme and try it with different values of \texttt{iterations}.
\end{document}
